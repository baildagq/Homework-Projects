\documentclass[]{article}
\usepackage{graphicx}
\usepackage{multirow}
\usepackage{ctex}
\usepackage{xeCJK}  
\usepackage{listings}
\usepackage{xcolor}
\usepackage{graphicx}
\usepackage{geometry}
\geometry{a4paper,left=3cm,right=3cm,top=2cm,bottom=2cm}
\lstset{
    language = VHDL,
    numbers=left, 
	breaklines=true,
    numberstyle= \tiny, 
    keywordstyle= \color{ blue!70},
    commentstyle= \color{red!50!green!50!blue!50}, 
    frame=shadowbox, % 阴影效果
    rulesepcolor= \color{ red!20!green!20!blue!20} ,
    escapeinside=``, % 英文分号中可写入中文
    xleftmargin=2em,xrightmargin=2em, aboveskip=1em,
    framexleftmargin=2em,
    tabsize=2
} 

\title{人工智能导论\\ 拼音输入法作业报告}
\author{齐强 2017011436 计75}


\begin{document}
\maketitle
\tableofcontents
\newpage
%abstract

\section{快速使用}
\begin{enumerate}
    \item note:以下命令需要在src文件夹下运行
    \item help: python3 main.py -h
    \item train: python3 --train --data \{指定训练数据path\}
    \item valid: python3 --input \{input file path\} --output \{output file path\}
\end{enumerate}

\section{文件结构}
\begin{enumerate}
    \item{src: 包含正式代码与自动测试相关代码
        \begin{itemize}
            \item common.py: 包含通用的设置,包括模型相关文件all-ch.txt, pinyin2ch.txt和模型的路径、总字数的数量等
            \item model.py: 包含Model类(代码主要部分),根据输入进行训练或预测
            \item main.py: 解析选项,决定训练或预测,解析inputfile,将一行的拼音转成list后调用model predict()预测,并将答案输出到outputfile
            \item train.sh: 训练脚本(新闻文件有点多,手动训练有点慢...)
        %    \item gen\_text.py: 从新闻训练数据中找出一定数量文本,并由句号、逗号分割产生粗糙训练文本
        %    \item gen\_py.py: 由粗糙训练文本产生拼音数据文件与正确答案文件
        %    \item valid.py: 根据main.py 产生的预测结果与gen\_py.py 产生的正确答案,得出模型准确率
        %    \item antutest.sh: 自动测试脚本
            \item autotest.sh | gen\_text.py | gen\_py.py | valid.py 自动测试相关
        \end{itemize}
    } 
    \item{data: 包含存储输入输入文件
    }
    \item{model: 包含模型相关文件
        \begin{itemize}
            \item all-ch.txt: 给定文件,确定所有进行统计的汉字
            \item pinyin2ch.txt: 给定文件,根据拼音确定相应可能汉字
            \item dict.json: 用于存储训练时与测试时需要用到的常用dict (省去部分准备时间)
            \item model.npz: 模型文件,采用numpy的数据存储格式,包含1-gram, 2-gram, 3-gram 组的出现频数
        \end{itemize}

    }
\end{enumerate}

\section{算法思路}
\begin{enumerate}
    \item 最终实现了字的二元、三元相结合的算法,同时保留了单二元的方法
    \item {
        train part:
        \begin{itemize}
            \item 遍历给定数据,统计所有单字出现的频数与所有2元组出现的频数,将结果分别保存在c1, c2数组中
            \item 取出c2数组中记录下的前 6763 高频的2元组,重新遍历训练数据,统计以这些高频2元组+其他字做3元组的出现频数
        \end{itemize}
    }
    \item {
        predict part:
        \begin{itemize}
            \item{
                predict(): 2元预测\\
                $
                    prob_{i,j} = max(prob_{i-1,k} + log(N(\omega_{i,j}\omega_{i-1,k})) - log(N(\omega_{i-1,k})))
                $
            }
            \item{
                predict2(): 2元 + 3元预测\\
                $if\ \omega_{lk}\  in\  [high frequency 2-gram word]:$\\
                \ \ \ \ $prob_{i, kj} = max(prob_{i-1, lk} + log(N(\omega_{i-2,l}\omega_{i-1,k}\omega_{i,j})) - log(N(\omega_{i-2,l}\omega_{i-1,k}))) $\\
                $else$\\
                \ \ \ \ $prob_{i, kj} = max(prob_{i-1, lk} + log(\omega_{i-1,k}\omega_{i,j})) - log(N(\omega_{i-1,k}))) $\\

            }
        \end{itemize}
    }
\end{enumerate}

\section{测试效果}
写了一些自动测试代码,从新闻训练数据中找出一定数量文本,并由句号、逗号分割产生粗糙训练文本,由粗糙训练文本产生拼音数据文件与正确答案文件,根据main.py 产生的预测结果与gen\_py.py 产生的正确答案进行对比,得出模型准确率

\begin{center}
    \begin{tabular}{lllll}
    \hline
    seq & total line & accuracy line & total word & accuracy word \\ \hline
     0 & 131& 0.2977& 1885 & 0.8233\\ \hline 
     1 & 153& 0.1699& 2266 & 0.7691\\ \hline
     2 & 196& 0.2102& 3059 & 0.8394\\ \hline
     3 & 167& 0.2455& 2135 & 0.8018\\ \hline
     4 & 100& 0.2300& 1436 & 0.7994\\ \hline
     5 & 212& 0.2783& 2304 & 0.8085\\ \hline
    \end{tabular}
\end{center}

\section{总结收获}
\begin{enumerate}
    \item{问题:
        \begin{itemize}
            \item 无法对多音字进行判断,导致训练时会存在偏差
            \item 因为时间有些来不及,没有更好的调参,对于平滑处理等不是很好
        \end{itemize}    
    }
    \item{
        总结:
        \begin{itemize}
            \item 2元与3元结合能够提升一些名词较长的句子的准确率
            \item 训练语料库与测试集的相关性,对于准确率有着非常大的影响,增加训练语料库会提高一定的正确率
        \end{itemize}
    }
\end{enumerate}


\end{document}
